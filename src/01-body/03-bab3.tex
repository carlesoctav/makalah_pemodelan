%-----------------------------------------------------------------------------%
\chapter{Penutup}
%-----------------------------------------------------------------------------%



%-----------------------------------------------------------------------------%
\section{Kesimpulan}
Dari hasil pengujian bab 2.5, dapat dilihat bahwa  tidak semua variabel \textit{proxy}  dapat menjelaskan variabilitas \textit{total colummn ozone} di setiap garis daerah (tropis, subtropis selatan, subtropis utara, kutub utara, kutub selatan). Pada daerah tropis, ditemukan bahwa variabel \textit{proxy} EESC dan \textit{aerosol backscatter ratio} adalah variabel \textit{proxy} yang dapat menjelaskan variabilitas TCO dengan baik. Di lain sisi, pada daerah subtropis selatan, EESC dan QBO-30 lah merupaka variabel penjelas yang baik. Namun, pada daerah subtropis utara, kutub selatan, kutub utara, tidak ditemukan variabel \textit{proxy} yang dapat menjelaskan variabilitas total column ozone dengan baik. 
%-----------------------------------------------------------------------------%

%-----------------------------------------------------------------------------%
\section{Saran}
Proses pembentukan dan penghancuran  lapisan ozon merupakan proses yang dinamik dan rumit. Kelima variabel \textit{proxy} yang digunakan belum dapat meng-\textit{capture} semua proses dinamik tersebut secara global. Oleh karena itu, pada penelitian selanjutnya, kami dapat mempertimbangkan variabel-variabel \textit{[proxy} lainnya, sehingga  memperluas ukuran model. Dengan begitu, kami harapkan bahwa performa model akan jauh lebih baik. Selain itu, rentang waktu tahun  (derajat bebas) dari  model juga dapat  diperpanjang, sehingga kami harapkan juga performa model yang lebih baik.
%-----------------------------------------------------------------------------%

