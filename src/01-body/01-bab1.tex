%-----------------------------------------------------------------------------%
\chapter{Pendahuluan}
%-----------------------------------------------------------------------------%
%-----------------------------------------------------------------------------%
\section{Latar Belakang}
%-----------------------------------------------------------------------------%
Lapisan ozon adalah wilayah stratosfer Bumi yang mengandung konsentrasi ozon (O3) tinggi dibandingkan di lapisan lain. Lapisan ozon utamanya ditemukan di bagian bawah stratosfer, dari sekitar 15 hingga 35 kilometer (9 hingga 22 mil) di atas Bumi. Ketebalan lapisan ozon bervariasi secara musiman dan geografis.

Lapisan ozon berperan penting sebagai penyerap radiasi ultraviolet Matahari. Lapisan ozon menyerap 97 hingga 99 persen sinar ultraviolet frekuensi menengah Matahari (dari panjang gelombang sekitar 200 nm hingga 315 nm). Jika tidak diserap, sinar ini berpotensi merusak kehidupan di permukaan Bumi.

\textit{Ozone depletion} adalah fenomena penipisan (pemecahan) ozon pada lapisan ozon di bumi yang disebabkan oleh \textit{ozone depleting substances}. \textit{Ozone-depleting substances} (ODS) adalah gas halogen yang memuat klorin dan/atau bromin yang mempunyai potensi untuk memecah ozon di stratosfer. Beberapa contoh dari ODS antara lain klorofluorokarbon (CFC), hidroklorofluorokarbon (HCFC), metil klorida, metil bromida, dan halon. \textit{Ozone-depleting substances} dapat disebabkan secara natural dan antropogenik.

Proses yang dapat memengaruhi variasi konsentrasi lapisan ozon antara lain, \textit{Midlatitude Halogen Chemistry}, \textit{Aerosol Effects}, \textit{Quasi-Biennial Oscillation}, \textit{Solar Cycle}.



%-----------------------------------------------------------------------------%
\section{Permasalahan}
%-----------------------------------------------------------------------------%
Pada bagian ini akan dijelaskan mengenai rumusan masalah 
yang \saya~hadapi dan ingin diselesaikan serta asumsi dan batasan 
yang digunakan dalam menyelesaikannya.


%-----------------------------------------------------------------------------%
\subsection{Rumusan Masalah}
%-----------------------------------------------------------------------------%
\begin{enumerate}
    \item Apa saja faktor-faktor yang memengaruhi variasi konsentrasi lapisan ozon?
    \item Bagaimana model matematis untuk mengkuantifikasi pengaruh masing-masing faktor terhadap variasi konsentrasi lapisan ozon?
\end{enumerate}


%-----------------------------------------------------------------------------%
\subsection{Batasan Permasalahan}
%-----------------------------------------------------------------------------%
Pada penelitian ini penulis tidak menggunakan model untuk prediksi, melainkan penulis hanya memodelkan untuk interpretasi supaya dapat menentukan faktor-faktor yang memengaruhi variasi konsentrasi lapisan ozon.


%-----------------------------------------------------------------------------%
\section{Tujuan}
%-----------------------------------------------------------------------------%
\begin{enumerate}
    \item Mengetahui faktor-faktor yang memengaruhi variasi konsentrasi lapisan ozon
    \item Membangun model matematis untuk mengkuantifikasi pengaruh masing-masing faktor terhadap variasi konsentrasi lapisan ozon
\end{enumerate}

%-----------------------------------------------------------------------------%
\section{Metodologi Penelitian}
%-----------------------------------------------------------------------------%
Metodologi penelitian yang penulis gunakan adalah metode kuantitatif dengan desain deskriptif.

