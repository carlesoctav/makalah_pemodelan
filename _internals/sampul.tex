%
% Sampul Laporan

%
% @author  unknown
% @version 1.01
% @edit by Andreas Febrian
%

\begin{titlepage}
    \begin{center}    
        \begin{figure}
            \begin{center}
                \includegraphics[width=2.5cm]{_internals/makara.eps}
            \end{center}
        \end{figure}    
        \vspace*{0cm}
        \bo{
        	UNIVERSITAS INDONESIA\\
        }
        
        \vspace*{1.0cm}
        % judul thesis harus dalam 14pt Times New Roman
        \bo{ANALISIS FAKTOR-FAKTOR PENGARUH VARIASI KONSENTRASI
        LAPISAN OZON STRATOSFER DENGAN REGRESI} \\[1.0cm]

        \vspace*{2.5 cm}    
        % harus dalam 14pt Times New Roman
        \bo{LAPORAN AKHIR KULIAH PEMODELAN MATEMATIS \\ SEMESTER GANJIL 2022} \\
        [1.0cm]

        \vspace*{3 cm}       
        % penulis dan npm
       \bo{ Agustinus Bravy Tetuko Ompusunggu​ (2006521300)\\
Antonius Rangga Hapsoro Wicaksono​ (2006568790​)\\
Carles Octavianus​ (2006568613)\\
Fenny Fadhilah Zakiyyatunnisa (2006568853)\\
Rafaella Garcinia Gayatri (2006572226)\\
Yohanes Bryan Sagala (2006568696) \\
}

        \vspace*{2.5cm}

        % informasi mengenai fakultas dan program studi
        \bo{
        	FAKULTAS MATEMATIKA DAN ILMU PENGETAHUAN ALAM
        	PROGRAM STUDI MATEMATIKA \\
        	DEPOK \\
        	2023
        }
    \end{center}
\end{titlepage}
